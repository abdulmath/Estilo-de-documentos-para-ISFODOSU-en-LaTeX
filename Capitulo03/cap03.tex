%-----------------------------------------------------------------------
% Capítulo 3
%-----------------------------------------------------------------------
\chapter{Título del capítulo}\label{cap:03}
\pagecolor{white}
\BgThispage
\thispagestyle{empty}
\pagestyle{isfodosuCD}
\epigraph{La libertad está en ser dueños de nuestra propia vida.}{Platón}

\lettrine[lraise=0, lines=4, loversize=0]{\textcolor{azulF}{P}}{ara} la lógica de proposiciones, la \textit{lógica 
formal} puede determinar la validez de cualquier razonamiento, donde la proposición analizada se deriva de una 
proposición sin analizar de otra u otras proposiciones que no se analizaban. En otras palabras: la lógica formal, a 
nivel de lógica de proposiciones, solo puede analizar formalmente de manera completa aquellos argumentos cuya estructura 
interna de las proposiciones que las componen en su validez no tiene ningún papel.

%-----------------------------------------
% \cleardoublepage
\subchapter{Problemas propuestos}
\pagestyle{probprop}
\pagecolor{paginaprob}
