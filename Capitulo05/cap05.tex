%-----------------------------------------------------------------------
% Capítulo 5
%-----------------------------------------------------------------------
\chapter{Título del capítulo}\label{cap:05}
\pagecolor{white}
\BgThispage
\thispagestyle{empty}
\pagestyle{isfodosuCD}
\epigraph{Se mide la inteligencia de un individuo por la cantidad de incertidumbres que es capaz de soportar.}{Immanuel 
Kant}

\lettrine[lraise=0, lines=4, loversize=0]{\textcolor{azulF}{E}}{n} este capítulo desarrollaremos el concepto de 
\textit{relaciones entre conjuntos}. Este es un concepto importante en matemáticas. Para nosotros, su mayor utilidad es 
que se basa en la definición de funciones que veremos más adelante. Ya estamos familiarizados con las relaciones, por 
ejemplo: $a=b$, la relación de \textit{igualdad}; $a<b$, la relación de \textit{menor que}; $X\subseteq Y$ la relación 
de subconjunto; $m|n$ la relación de \textit{divisor de}, y así muchas otras.

%-----------------------------------------
% \cleardoublepage
\subchapter{Problemas propuestos}
% \setenumerate[1]{label=\bfseries{\alph*)\quad}, labelindent=\parindent}
% \setenumerate[2]{label=\bfseries\arabic*.}
% \setenumerate[3]{label=\bfseries{\roman*})}
\pagestyle{probprop}
\pagecolor{paginaprob}

