%-----------------------------------------------------------------------
% Capítulo 2
%-----------------------------------------------------------------------
\chapter{Título del capítulo}\label{cap:02}
\pagecolor{white}
\BgThispage
\thispagestyle{empty}
\pagestyle{isfodosuCD}
\epigraph{Nunca se alcanza la verdad total, ni nunca se está totalmente alejado de ella.}{Aristóteles}

\lettrine[lraise=0, lines=4, loversize=0]{\textcolor{azulF}{A}}{lgunas} disciplinas, como la música y las matemáticas, 
han recurrido al lenguaje artificial o simbólico para simplificar la expresión de ideas complejas. A diferencia de los 
lenguajes naturales como el español, el inglés o el ciguayo, que son herramientas generales de comunicación, los 
lenguajes simbólicos están diseñados con un propósito específico, lo que nos permite expresar pensamientos en términos 
precisos sin ambigüedad y contenido emocional.



\textbf{Forma de usar el ambiente del ejemplo con nombre del ejemplo. Recordar que cada ambiente tiene dos copciones, 
la primero es para el nombre y la segunda para la etiqueta de referencia.}
\begin{ejem}[\;\;(Negación)][ejem:02:03]
\centering
La función seno es periódica.

\smallskip

La función seno \textcolor{Naranja}{\textbf{no}} es periódica.
\end{ejem}


%-----------------------------------------
\subsection{\textit{Título de la subsección}}

\textbf{Ejemplo de una tabla con encabezado gris}

\begin{table}[H]
\begin{tabular}{c|c|c}
\rowcolor{gray!20}$p$ & $q$ & $p\wedge q$\\
\toprule
\textcolor{blue}{\textbf{V}} & \textcolor{blue}{\textbf{V}} & \textcolor{blue}{\textbf{V}}\\
\textcolor{blue}{\textbf{V}} & \textcolor{red}{\textbf{F}} & \textcolor{red}{\textbf{F}}\\
\textcolor{red}{\textbf{F}} & \textcolor{blue}{\textbf{V}} & \textcolor{red}{\textbf{F}}\\
\textcolor{red}{\textbf{F}} & \textcolor{red}{\textbf{F}} & \textcolor{red}{\textbf{F}}\\
\toprule
\end{tabular}
\caption{Tabla de verdad para la conjunción}\label{tab:02:03}
\end{table}

\textbf{Ejemplo de una tabla con encabezado azul claro}
\begin{table}[H]
\begin{tabular}{c|c|c|c|c|c}
\rowcolor{LightBlue2}$p$ & $q$ & $p\wedge q$ & $p\longrightarrow q$ & $\neg(p\longrightarrow q)$ & $(p\wedge 
q)\vee\neg(p\longrightarrow q)$\\
\toprule
\textcolor{red}{\textbf{F}} & \textcolor{red}{\textbf{F}} & \textcolor{red}{\textbf{F}} & \textcolor{blue}{\textbf{V}} & 
\textcolor{red}{\textbf{F}} & \textcolor{red}{\textbf{F}}\\
\textcolor{red}{\textbf{F}} & \textcolor{blue}{\textbf{V}} & \textcolor{red}{\textbf{F}} & \textcolor{blue}{\textbf{V}} 
& \textcolor{red}{\textbf{F}} & \textcolor{red}{\textbf{F}}\\
\textcolor{blue}{\textbf{V}} & \textcolor{red}{\textbf{F}} & \textcolor{red}{\textbf{F}} & \textcolor{red}{\textbf{F}} & 
\textcolor{blue}{\textbf{V}} & \textcolor{blue}{\textbf{V}}\\
\textcolor{blue}{\textbf{V}} & \textcolor{blue}{\textbf{V}} & \textcolor{blue}{\textbf{V}} & 
\textcolor{blue}{\textbf{V}} & \textcolor{red}{\textbf{F}} & \textcolor{blue}{\textbf{V}}\\ 
\toprule
\end{tabular}
\caption{Tabla del ejemplo \ref{ejem:02:46}}\label{tab:02:07}
\end{table}


%-----------------------------------------
\cleardoublepage
\subchapter{Problemas propuestos}
\setenumerate[1]{label=\bfseries{\alph*)\quad}, labelindent=\parindent}
\setenumerate[2]{label=\bfseries\arabic*.}
\setenumerate[3]{label=\bfseries{\roman*})}
\pagestyle{probprop}
\pagecolor{paginaprob}

\begin{prob} ¿Cuáles de las siguientes son proposiciones? En caso que sea un proposición, diga si es verdadera o falsa.
\begin{multicols}{2}
\begin{enumerate}
     \item $7-4=3$.
     \item $5^{4}<3^{2}$.
     \item En la escuela nos enseñan a caminar.
     \item Tu voto es tu opinión.
     \item ¿Te duele?
     \item !`Cállate la boca!
     \item Aquel árbol es azul.
     \item La música es una expresión del arte.
\end{enumerate}
\end{multicols}
\end{prob}

