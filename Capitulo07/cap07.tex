
%-----------------------------------------------------------------------
% Capítulo 7
%-----------------------------------------------------------------------
\chapter{Título del Capítulo}\label{cap:07}
\pagecolor{white}
\BgThispage
\thispagestyle{empty}
\pagestyle{isfodosuCD}
\epigraph{Los educadores, más que cualquier otra clase de profesionales, son los guardianes de la 
civilización.}{Bertrand Russell}

\lettrine[lraise=0, lines=4, loversize=0]{\textcolor{azulF}{E}}{n} este capítulo trataremos la numerabilidad o no de 
conjuntos, en particular se demostrará la numerabilidad de $\mathbb{Q}$ y la no numerabilidad de $\mathbb{R}$. Pero 
antes debemos introducir algunos conceptos previos sobre la cardinalidad de conjuntos. El tamaño de un conjunto finito 
puede medirse fácilmente; por ejemplo, el tamaño del conjunto $A=\{1,2,3,\ldots,50\}$ es $50$ porque tiene $50$ 
elementos, y el tamaño de los conjuntos $B=\{\pi,2,30,-2\}$ y $C=\{9,11,-1,5\}$ es $4$.

\textbf{ejemplo del ambiente para Lemas}

\begin{lma}[][lema:07:01] El conjunto de números racionales positivos, $\mathbb{Q}^{+}$, es numerable.
\end{lma}

%-----------------------------------------
\subchapter{Problemas propuestos}
% \setenumerate[1]{label=\bfseries{\alph*)\quad}, labelindent=\parindent}
% \setenumerate[2]{label=\bfseries\arabic*.}
% \setenumerate[3]{label=\bfseries{\roman*})}
\pagestyle{probprop}
\pagecolor{paginaprob}

