\chapter*{\notationname}
\pagestyle{especial}
\chaptermark{\notationname}
\phantomsection
\addcontentsline{toc}{listasf}{\notationname}
%\section*{Notación}
%\begin{table}[htbp]
\begin{longtable}{p{2cm}p{9cm}}
$\mathbb{N}$ & Conjunto de los números naturales \\
$\mathbb{N}^\ast$ & Conjunto de los números naturales extendidos (incluyen al cero)\\
$\mathbb{Z}$ & Conjunto de los números enteros \\
$\mathbb{Z}^+$ & Conjunto de los números enteros positivos\\
$\mathbb{Z}^-$ & Conjunto de los números enteros negativos\\
$\mathbb{Q}$ & Conjunto de los números racionales \\
$\mathbb{Q}^+$ & Conjunto de los números racionales positivos\\
$\mathbb{Q}^-$ & Conjunto de los números racionales negativos\\
$\mathbb{R}$ & Conjunto de los números reales \\
$\mathbb{C}$ & Conjunto de los números complejos \\
$\mathcal{U}$ & Conjunto universal \\
$\emptyset$ & Conjunto vacío \\
$\infty$ & infinito\\
$\in$ & pertenece a \\
$\notin$ & no pertenece a \\
$<$ & menor que \\
$>$ & mayor que \\
$\nless$ & no es menor que \\
$\ngtr$ & no es mayor que \\
$\le$ ó $\leq$ & menor o igual que \\
$\geqslant$ ó $\geq$& mayor o igual que \\
$\nleqslant$ ó $\nleqq$ & no es menor o igual que \\
$\ngeqslant$ ó $\ngeqq$& no es mayor o igual que \\
$=$ & igual \\
$\not=$ & no es igual a \\
$\subset$ & subconjunto de \\
$\subseteq$ & subconjunto o igual a \\
$\not\subset$ & no es subconjunto de \\
$\not\subseteq$ & no es subconjunto o igual a \\
$a|b$ & $a$ es un divisor de $b$\\
$aRb$ ó $a\sim b$ & $a$ está relacionado con $b$\\
$\sim$ & relación \\
$a\cancel{R}b$ ó $a\not\sim b$ & $a$ no está relacionado con $b$\\
$\not\sim$ & no es una relación \\
$[n]_{\sim}$ ó $[n]$ & clase de equivalencia de $n$ con respecto a la relación $\sim$\\
$\preccurlyeq$ & relación de orden parcial \\
$\prec$ & relación de orden estricto \\
$\operatorname{dom}(R)$ & dominio de $R$ \\
$\operatorname{ran}(R)$ & rango de $R$\\
$\circ$ & composición \\
$\iota(\cdot)$ & función identidad \\
$\neg$ & negación de \\
$\neg\neg$ & doble negación de \\
$\vee$ & disyunción \\
$\wedge$ & conjunción \\
$\longrightarrow$ ó $\Longrightarrow$ & implicación \\
$\longleftrightarrow$ ó $\Longleftrightarrow$ & bicondicional \\
$\square$ & fin de la solución \\
$\blacksquare$ & fin de la demostración \\
$\equiv$ & equivalente \\
$\forall$ & cuantificador universal (para todo) \\
$\exists$ & cuantificador existencial (existe) \\
$\nexists$ & negación del cuantificador existencial (no existe) \\
$^\circ C$ & grados Celcius\\
$\cos(\frac{\pi}{3})$ & la función coseno evaluado en $\frac{\pi}{3}$ \\
$\alpha$ & letra griega alfa \\
$\beta$ & letra griega beta \\
$\theta$ & letra griega theta \\
$\psi$ & letra griega psi \\
$\phi$ & letra griega phi \\
$\varphi$ & letra griega varphi \\
$\rho$ & letra griega rho \\
$\pi$ & letra griega pi, usada para denotar el número trascendente $\pi$ \\
$e$ & número trascendente $e$ \\
$\times$ ó $\cdot$ & producto \\
$:$ ó $/$ & tal que \\
$\{\quad\}$ & para denotar conjuntos \\
$\mathcal{P}(X)$ & conjunto pontencia de $X$ ó partes de $X$ \\ 
$\cup$ & unión \\
$\cap$ & intersección \\
$\setminus$ ó $-$ & diferencia de conjunto \\
$\vartriangle$ & diferencia simétrica \\
$A\times B$ & producto cartesiano de $A$ con $B$ \\
$\mathbb{R}^2$ & plano cartesiano o bidimensional \\
$A^c$ & complemento del conjunto $A$ \\
$|x|$ & valor absoluto de $x$ \\
$|A|$ & cardinal o cantidad de elementos del conjunto $A$ \\
\end{longtable}
% \newpage
%\end{table}
%


%\phantomsection
%\addcontentsline{toc}{listasf}{Acrónimos}
%\section*{Acrónimos}
%\begin{table}[htbp]
%\begin{tabular}{p{2cm}p{10cm}}
%Escuela Técnica Superior de In
%LTI & Lineal Invariante con el Tiempo \\
%LTV& Lineal Variable con el Tiempo\\
%AWGN& Ruido blanco gaussiano aditivo\\
%DMS& Fuente discreta sin memoria\\
%AEP& Propiedad de equipartición asintótica\\
%WLLN& Ley Débil de los Grandes Números\\
%DMC& Canal Discreto sin Memoria\\
%BSC& Canal Simétrico Binario\\
%BEC& Canal Binario con Borrado\\
%\end{tabular}
%\end{table}


%\nota{El libro de Lapidoth tiene una excelente recopilación.}
