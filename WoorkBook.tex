%:Clase del documento
\documentclass[fontsize=10pt, Myfinal=false, twoside, numbers=noenddot, x11names]{scrbook}
%, Myfinal=true, Minion=true, English=true

% Pie de pagina configuracion
% \usepackage{fancyhdr}
% \pagestyle{fancy}
% \rfoot[]{\includegraphics{figuras/LetrasISFODOSU.png}}

%:Paquete de estilos propuesto
\usepackage{libroISFODOSU}

%:Paquete específico para cargar tikz (y sus librerías) y pgfplots
\usepackage{dtsc-creafig}

%:Paquete para notaciones específicas
\usepackage{notacion}

%:Paquete para incorporar aspectos concretos de la edición
\usepackage{edicionLibro}

%:Estas líneas de código son INNECESARIAS excepto para mostrar determinadas características en este manual. Pueden eliminarse o comentarse sin ningún problema.
%Se usan para compilar el capítulo estilolibroetsi.tex
\usepackage[final]{showexpl}
\lstset{explpreset={frame=none,rframe={}, numbers=none,numbersep=3pt, columns=flexible,language={[LaTeX]TeX},basicstyle=\ttfamily,keywordstyle=\color{blue}}}%numberstyle=\tiny,

\makeatletter
\patchcmd{\SX@codeInput}{xleftmargin=0pt,xrightmargin=0pt}{}
  {\typeout{***Successfully patched \protect\SX@codeInput***}}
  {\typeout{***ERROR! Failed to patch \protect\SX@codeInput***}}
\makeatother
%:Hasta AQUI

%:Para modificar fácilmente la fuente del texto. 
\makeatletter
\ifdtsc@Minion % Queremos utilizar la fuente Minion y lo hemos declarado al principio
	\ifluatex
		\setmainfont[Renderer=Basic, Ligatures=TeX,	% Fuente del texto 
		Scale=1.01,
		]{Minion Pro}
   		% En este caso conviene modificar ligeramente el tamaño de las fuentes matemáticas
		\DeclareMathSizes{10}{10.5}{7.35}{5.25}
		\DeclareMathSizes{10.95}{11.55}{8.08}{5.77}
		\DeclareMathSizes{12}{12.6}{8.82}{6.3}
%		\setmainfont[Renderer=Basic, Ligatures=TeX,	% Fuente del texto 
%		]{Adobe Garamond Pro}
%		\setmainfont[Renderer=Basic, Ligatures=TeX,	% Fuente del texto 
%		]{Palatino LT Std}
	\fi
\else
	\ifluatex
		% Para utilizar la fuente Times New Roman, o alguna otra que se tenga instalada
		\setmainfont[Renderer=Basic, Ligatures=TeX,	% Fuente del texto 
		Scale=1.0,
		]{Times New Roman}
	\else
		\usepackage{tgtermes} 	%clone of Times
		%\usepackage[default]{droidserif}
		%\usepackage{anttor} 	
	\fi
\fi
\makeatother

%Por si quieren usar bibliografía con BIBER
%BIBER%%:Para la bibliografía en BIBER, descomentar las líneas siguientes
%\defbibheading{etsi}[]{%
%	\chapter*{Bibliografía}%
%	\chaptermark{Bibliografía} 
%	\markboth{#1}{#1}}
%\addbibresource{bibliografiaLibroETSI.bib}

% Ejemplo de Glosario
\newacronym[type=main]{ISFODOSU}{ISFODOSU}{Instituto Superior de Formación Docente Salomé Ureña}
\newacronym[type=main]{EPH}{EPH}{Recinto ``XXXXX CCCCCCC VVVVVV''}

\makeindex
% \makeglossaries %Si no se quiere el glosario, comentar esta línea.

%TAMAÑO LIBRO O A4
%Para definir el tamaño del documento, hay que elegir uno de los siguientes y comentar el otro
%Formato Libro
\geometry
{paperheight=254mm,%
paperwidth=177.8mm,%
top=25mm,%
headsep=7.5mm,%
footskip=10mm,%
textheight=210mm,%
textwidth=132mm,%
bindingoffset=15mm,%
twoside}

% \usepackage[a4,cross,center]{crop}% para poner las cruces de esquina de página, poner la opción cross. Comentar para 
%que el fichero pdf se genere con el tamaño deseado

%:Esquema de numeración por defecto
\setenumerate[1]{label=\normalfont\bfseries{\arabic*.}, leftmargin=*, labelindent=\parindent}
\setenumerate[2]{label=\normalfont\bfseries{\alph*}), leftmargin=*}
\setenumerate[3]{label=\normalfont\bfseries{\roman*.}, leftmargin=*}
\setlist{itemsep=.1em}
\setlength{\parindent}{1.0 em}

\setcounter{tocdepth}{5}						% El nivel hasta el que se muestra el índice 

%PARA ENVIAR MANUSCRITO
%Para manuscrito a revisar, descomentar siguiente línea y así se pondrá el interlineado a 1.5 líneas
%\onehalfspacing  %necesario setspace package
%
%
%Para otros espaciados:
%  Type \singlespacing, \onehalfspacing, or \doublespacing, depending on the spacing you want.
%  Type \setstretch{x} where x is a number indicating the spacing you want.
%
 %  For example, the command \setstretch{3} produces triple spacing.
%
% \let\cleardoublepage\clearpage
\setlength{\marginparwidth}{90pt}
% \setlength{columnsep}{3cm}

%:Empieza el documento
\hypersetup{pageanchor=false}
\begin{document}

%PORTADA
%ver edicionLibro.sty para modificaciones

%:Para crear la portada y la portada interior (pagina titular)
\titulo{Título del Libro}
\subtitulo{Serie Matemáticas} %Se puede usar en libro
\edicion{Primera edición} %Se puede usar en libro

\autor{Autor o autores del libro}

\universidad{Instituto Superior de Formación Docente Salomé Ureña}
% \centro{Recinto Félix Evaristo Mejía}
\centro{Aquí el recinto al cual pertenece el autor}
\departamento{Vicerrectoría Académica}

\fecha{Año de publicación}

% Para establecer palabras claves en el fichero pdf
\hypersetup
	{
 	linkcolor=black, %Tocar para poner color en enlaces
	pdfauthor={\elautor},
	pdftitle={\eltitulo}, 
	citecolor=black, %Tocar para poner color en enlaces, eg refcolor o blue
	pdfkeywords={Serie Matemática, Instituto Superior de Formación Docente Salomé Ureña}	
	 }

%:Construcción de la cubierta, páginas de título y copyright
%\cubiertalibro{figuras/imagenlibro.png} 

\paginatitulo

\paginatituloautor{figuras/isfodosu.pdf} %Logo del Dep

% \copyright

%:Todo lo que constituye la primera parte del libro que no es el cuerpo del libro en realidad
\frontmatter
\pagenumbering{Roman} %Pone la numeración en mayúscula (En español parece que es obligatorio)

%:La dedicatoria, si queremos ponerla
\dedicatoria{Aquí va la dedicatoria} 

\chapter*{Agradecimientos}
\pagestyle{especial}
\pagestyle{empty}
\chaptermark{Agradecimientos}
\phantomsection
\addcontentsline{toc}{listasf}{Agradecimientos}

\lettrine[lraise=0, lines=4, loversize=0]{\textcolor{azulF}{A}}{uí} van las palabras de agradecimientos, entre 
líneas se usa una salto de parrafo, para que no quede muy pegado los parrafos. El salto puede ser de diferentes tipos:
\begin{enumerate}
     \item \verb|\smallskip|
     \item \verb|\medskip|
     \item \verb|\bigskip|
\end{enumerate}


{\flushleft{\hfill \emph{Fulanito}}}%
\vspace{-.3cm}
% {\flushleft{\hfill \emph{Instituto Superior de Formación Docente Salomé Ureña}}}%
{\flushleft{\hfill \emph{Santo Domingo, 2022}}}%


%PFC/PFM/TESIS
% \include{resumen/resumen} %Descomentar para proyectos/tesis
\chapter*{Prefacio}
\pagestyle{especial}
\chaptermark{Prefacio}
\phantomsection
\addcontentsline{toc}{listasf}{Prefacio}
\thispagestyle{empty}

%\lettrine[lraise=0.7, lines=1, loversize=-0.25]{E}{n} %Para arriba
\lettrine[lraise=0, lines=4, loversize=0]{\textcolor{azulF}{E}}{ste} texto que se coloca es el prefacio que bien 
pudiera ser escrito por el autor o alguna otra persona.

\medskip

{\flushleft{\hfill \textbf{Fulanito}}}%
\vspace{-.3cm}
{\flushleft{\hfill \emph{Instituto Superior de Formación Docente Salomé Ureña}}}%
{\flushleft{\hfill \emph{Santo Domingo, 2022}}}%
 %Comentar para proyectos/tesis


%Índice normal, el completo
\cleardoublepage
\phantomsection
% \addcontentsline{toc}{listasf}{Índice}
\pagestyle{especial}
\tableofcontents
% \thispagestyle{empty}

%:Indice de figuras, coméntese las siguientes líneas si no se desea
\cleardoublepage
\phantomsection

%:Para añadir una línea en blanco en el TOC y separar esta lista
\addtocontents{toc}{\protect\mbox{}\protect\hspace*{0pt}\par}
\addcontentsline{toc}{listasb}{\listfigurename}
\pagestyle{especial}
\listoffigures
\thispagestyle{empty}

%:Indice de tablas, coméntese las siguientes líneas si no se desea
\cleardoublepage
\phantomsection
\addcontentsline{toc}{listasb}{\listtablename}
\pagestyle{especial}
\listoftables
\thispagestyle{empty}

% %:Indice de Programas
% \cleardoublepage
% \phantomsection
% \addcontentsline{toc}{listasb}{\lstlistlistingname}
% \pagestyle{especial}
% \lstlistoflistings

\chapter*{\notationname}
\pagestyle{especial}
\chaptermark{\notationname}
\phantomsection
\addcontentsline{toc}{listasf}{\notationname}
%\section*{Notación}
%\begin{table}[htbp]
\begin{longtable}{p{2cm}p{9cm}}
$\mathbb{N}$ & Conjunto de los números naturales \\
$\mathbb{N}^\ast$ & Conjunto de los números naturales extendidos (incluyen al cero)\\
$\mathbb{Z}$ & Conjunto de los números enteros \\
$\mathbb{Z}^+$ & Conjunto de los números enteros positivos\\
$\mathbb{Z}^-$ & Conjunto de los números enteros negativos\\
$\mathbb{Q}$ & Conjunto de los números racionales \\
$\mathbb{Q}^+$ & Conjunto de los números racionales positivos\\
$\mathbb{Q}^-$ & Conjunto de los números racionales negativos\\
$\mathbb{R}$ & Conjunto de los números reales \\
$\mathbb{C}$ & Conjunto de los números complejos \\
$\mathcal{U}$ & Conjunto universal \\
$\emptyset$ & Conjunto vacío \\
$\infty$ & infinito\\
$\in$ & pertenece a \\
$\notin$ & no pertenece a \\
$<$ & menor que \\
$>$ & mayor que \\
$\nless$ & no es menor que \\
$\ngtr$ & no es mayor que \\
$\le$ ó $\leq$ & menor o igual que \\
$\geqslant$ ó $\geq$& mayor o igual que \\
$\nleqslant$ ó $\nleqq$ & no es menor o igual que \\
$\ngeqslant$ ó $\ngeqq$& no es mayor o igual que \\
$=$ & igual \\
$\not=$ & no es igual a \\
$\subset$ & subconjunto de \\
$\subseteq$ & subconjunto o igual a \\
$\not\subset$ & no es subconjunto de \\
$\not\subseteq$ & no es subconjunto o igual a \\
$a|b$ & $a$ es un divisor de $b$\\
$aRb$ ó $a\sim b$ & $a$ está relacionado con $b$\\
$\sim$ & relación \\
$a\cancel{R}b$ ó $a\not\sim b$ & $a$ no está relacionado con $b$\\
$\not\sim$ & no es una relación \\
$[n]_{\sim}$ ó $[n]$ & clase de equivalencia de $n$ con respecto a la relación $\sim$\\
$\preccurlyeq$ & relación de orden parcial \\
$\prec$ & relación de orden estricto \\
$\operatorname{dom}(R)$ & dominio de $R$ \\
$\operatorname{ran}(R)$ & rango de $R$\\
$\circ$ & composición \\
$\iota(\cdot)$ & función identidad \\
$\neg$ & negación de \\
$\neg\neg$ & doble negación de \\
$\vee$ & disyunción \\
$\wedge$ & conjunción \\
$\longrightarrow$ ó $\Longrightarrow$ & implicación \\
$\longleftrightarrow$ ó $\Longleftrightarrow$ & bicondicional \\
$\square$ & fin de la solución \\
$\blacksquare$ & fin de la demostración \\
$\equiv$ & equivalente \\
$\forall$ & cuantificador universal (para todo) \\
$\exists$ & cuantificador existencial (existe) \\
$\nexists$ & negación del cuantificador existencial (no existe) \\
$^\circ C$ & grados Celcius\\
$\cos(\frac{\pi}{3})$ & la función coseno evaluado en $\frac{\pi}{3}$ \\
$\alpha$ & letra griega alfa \\
$\beta$ & letra griega beta \\
$\theta$ & letra griega theta \\
$\psi$ & letra griega psi \\
$\phi$ & letra griega phi \\
$\varphi$ & letra griega varphi \\
$\rho$ & letra griega rho \\
$\pi$ & letra griega pi, usada para denotar el número trascendente $\pi$ \\
$e$ & número trascendente $e$ \\
$\times$ ó $\cdot$ & producto \\
$:$ ó $/$ & tal que \\
$\{\quad\}$ & para denotar conjuntos \\
$\mathcal{P}(X)$ & conjunto pontencia de $X$ ó partes de $X$ \\ 
$\cup$ & unión \\
$\cap$ & intersección \\
$\setminus$ ó $-$ & diferencia de conjunto \\
$\vartriangle$ & diferencia simétrica \\
$A\times B$ & producto cartesiano de $A$ con $B$ \\
$\mathbb{R}^2$ & plano cartesiano o bidimensional \\
$A^c$ & complemento del conjunto $A$ \\
$|x|$ & valor absoluto de $x$ \\
$|A|$ & cardinal o cantidad de elementos del conjunto $A$ \\
\end{longtable}
% \newpage
%\end{table}
%


%\phantomsection
%\addcontentsline{toc}{listasf}{Acrónimos}
%\section*{Acrónimos}
%\begin{table}[htbp]
%\begin{tabular}{p{2cm}p{10cm}}
%Escuela Técnica Superior de In
%LTI & Lineal Invariante con el Tiempo \\
%LTV& Lineal Variable con el Tiempo\\
%AWGN& Ruido blanco gaussiano aditivo\\
%DMS& Fuente discreta sin memoria\\
%AEP& Propiedad de equipartición asintótica\\
%WLLN& Ley Débil de los Grandes Números\\
%DMC& Canal Discreto sin Memoria\\
%BSC& Canal Simétrico Binario\\
%BEC& Canal Binario con Borrado\\
%\end{tabular}
%\end{table}


%\nota{El libro de Lapidoth tiene una excelente recopilación.}
 %No incluir si no se quiere, comentándolo

%:Empieza el contenido del libro
\mainmatter

%:Página por defecto
\pagestyle{isfodosuCD}
\hypersetup{pageanchor=true}

%:Para incluir toda la referencia bibliográfica aunque no se cite, descomente la siguiente línea
%\nocite{*}
\nocite{UZ2011,UZC2011}

%:Los diferentes capítulos, en carpetas separadas
% \include{introduccion/introduccion}
%-----------------------------------------------------------------------
%	Capítulo 1
%-----------------------------------------------------------------------
\chapter{Título del Capítulo}\label{cap:01}
\BgThispage
\thispagestyle{empty}
\pagestyle{isfodosuCD}
\epigraph{No puedo enseñar nada a nadie. Solo puedo hacerles pensar.}{Sócrates}

\lettrine[lraise=0, lines=4, loversize=0]{\textcolor{azulF}{L}}{a} lógica tradicional como parte de la filosofía es una 
de las disciplinas científicas más antiguas. Se remonta a los estoicos y al filósofo, polímata y científico 
\textcolor{azulF}{\textbf{Aristóteles}}, y es la raíz de lo que hoy se llama \textbf{lógica filosófica}. Sin embargo, la 
lógica matemática es una disciplina relativamente joven, que surgió a partir de los esfuerzos de: 
\textcolor{azulF}{\textbf{Giuseppe Peano}}, matemático, lógico y filósofo italiano, \textcolor{azulF}{\textbf{Friedrich 
Ludwig Gottlob Frege}}, matemático, lógico y filósofo alemán y \textcolor{azulF}{\textbf{Bertrand Arthur William 
Russell}}, filósofo, matemático, lógico y escritor británico, que contribuyeron a reducir las matemáticas por completo a 
la lógica. A lo largo del siglo $XX$ se ha desarrollado de forma constante hasta convertirse en una amplia disciplina 
con varias subáreas y numerosas aplicaciones en matemáticas, informática, lingüística y filosofía. 

\medskip

\textbf{Este es un ejemplo de una caja resaltada en color gris}
\begin{caja}
\centering
\textit{el rigor lógico del razonamiento utilizado para justificar los resultados}.
\end{caja}

%-----------------------------------------
\section{Título de la sección}

Podemos definir las matemáticas como el estudio del número y del espacio. Aunque se pueden encontrar representaciones en el mundo físico, el objeto de las matemáticas no es físico. En cambio, los objetos matemáticos son abstractos, como las ecuaciones del álgebra o los puntos y las líneas de la geometría. Solo se encuentran como ideas en las mentes. Estas ideas conducen a veces al descubrimiento de otras ideas que no se manifiestan en el mundo físico, como cuando se estudian diversas magnitudes del infinito, mientras que otras conducen a la creación de objetos tangibles, como los puentes o los computadores.

\textbf{Un ejemplo de como colocar item con colores en azul con degradado}
\begin{itemize}[label=\ptom]
 \item De las categorías
 \item Tópicos 
 \item Refutaciones sofísticas
 \item Sobre la interpretación
 \item Primeros analíticos
 \item Segundos analíticos.
\end{itemize}

\textbf{Este es un ejemplo de definición:}
\begin{defi}[][def:01:01]
Una \textbf{deducción} es un discurso proveniente del latín \textit{logos} del cual, suponiendo ciertas cosas, resulta la necesidad de otra cosa diferente o distinta de las cosas supuestas, solo por haber sido supuestas estas cosas.
\end{defi}


\textbf{Un ejemplo de como usar las cajas de ejemplos:}
\begin{ejem}[][ejem:01:01] Supongamos nos dicen que un cierto número natural es menor que $35$, y además que el número en cuestión es divisible por $4$ y que al sumarle $3$ obtenemos un número divisible por $5$.
\end{ejem}


\textbf{Este es una ambiente definido en el estilo para las soluciones de los ejemplos.}

\begin{soln} ¿Podemos a partir de esta información inferir cuál o cuáles son los números?

\medskip

Ahora bien, como el número buscado es un número divisible por 4 y menor que 35, entonces el número debe ser alguno de los siguientes:
\[4,\quad 8,\quad 12, \quad 16, \quad 20, \quad 24, \quad 28, \quad \hbox{ó}\quad 32\]
adicionalmente, se pide que al número si le sumamos $3$, el número es divisible por $5$, entonces si exploramos tal situación, obtenemos los que podrían ser una posible solución, y son:
\[7,\quad 11,\quad 15, \quad 19, \quad 23, \quad 27, \quad 31, \quad 35,\]
luego entre estos números, los números $15$ y  $35$ son los únicos que son divisibles por $5$, por lo tanto, concluimos que los números buscados son $15$ y $35$.
\end{soln}

\textbf{Este es un ejemplo de items enumerados en color verde claro para los ejemplos:}
\begin{ejem}[][ejem:01:09] Los siguientes enunciados no son proposiciones:
\begin{enumerate}[label=\itembolasverdes{\arabic{*}}]
 \item\label{ejemplo:01:09:01} Espérame!
 \item\label{ejemplo:01:09:02} ¿Por qué estudias matemáticas?
 \item\label{ejemplo:01:09:03} $x+y=x$
 \item\label{ejemplo:01:09:04} ¡A estudiar!
 \item\label{ejemplo:01:09:05} Él es un estudiante.
\end{enumerate}
\end{ejem}

\textbf{Aquí pueden ver un ejemplo de como referenciar los item enumerados del ejemplo anterior:}
El enunciado \ref{ejemplo:01:09:03} no es una proposición, pues no hemos especificado el significado de los símbolos $x$ e $y$, y por esto no podemos decir si es verdadera o falsa. Sin embargo, si dijéramos lo siguiente
\[x+y=x\quad\mbox{ para algún }x,y\in\mathbb{Z}\]
entonces esa afirmación es una proposición verdadera. Pues tenemos, por ejemplo, que cuando $x=1$ y $y=0$ se cumple que $x+y=x$.


\textbf{Este es un ejemplo como usar el ambiente tabular para que las tablas queden con líneas más espaciadas:}
\begin{center}
\begin{tabular}{c}
\textcolor{azulF}{\textbf{Todos los $X$ son $Y$}}\\
\textcolor{azulF}{\textbf{Algunos $Z$ son $X$}}\\
\midrule
\textcolor{Naranja}{\textbf{Algunos $Z$ son $Y$}}
\end{tabular}
\end{center}

\textbf{Este es un entorno para colocar en una caja blanca con borde negro algunas citas importantes a modo de resumen 
en el texto, adicionalmente tambien puede ver como hacer enumeración de item en circulos azules para colocar fuera de 
los ambientes de ejemplos.}

\begin{cajablanca}
Al completar el argumento debemos propender para que sea el mejor argumento posible. Este requiere lo siguiente:
\begin{enumerate}[label=\itembolasazules{\arabic*}]
	\item la identificación de la(s) premisa(s) necesaria(s) para que el argumento sea deductivamente válido o al menos 
inductivamente fuerte; o
	\item la identificación de la conclusión que se sigue deductiva o inductivamente de las premisas dadas.
\end{enumerate}
\end{cajablanca}


%-----------------------------------------
\cleardoublepage
\subchapter{Problemas propuestos}
% \thepagestyle{empty}
\setenumerate[1]{label=\bfseries{\alph*)\quad}, labelindent=\parindent}
\setenumerate[2]{label=\bfseries\arabic*.}
\setenumerate[3]{label=\bfseries{\roman*})}
\pagestyle{probprop}
\pagecolor{paginaprob}
% \captionsetup[figure]{textformat=simple}

\begin{prob} Indique cuáles de las siguientes oraciones son enunciados.
\begin{enumerate}
 \item ?`Dónde queda Samarcanda?
 \item !`No me vuelvas a llamar!
 \item Beijing es una ciudad enorme.
 \item Por favor, cierra la puerta cuando salgas.
 \item La autosuficiencia petrolera del país solo durará cinco años más.
 \item ?`Quién va primero?
 \item $4+6=10$
 \item Este ejercicio es corto.
\end{enumerate}
\end{prob}

\textbf{Bajo el ambiente prob se pueden generar los ejercicios de cada capítulo del libro.}

%-----------------------------------------------------------------------
% Capítulo 2
%-----------------------------------------------------------------------
\chapter{Título del capítulo}\label{cap:02}
\pagecolor{white}
\BgThispage
\thispagestyle{empty}
\pagestyle{isfodosuCD}
\epigraph{Nunca se alcanza la verdad total, ni nunca se está totalmente alejado de ella.}{Aristóteles}

\lettrine[lraise=0, lines=4, loversize=0]{\textcolor{azulF}{A}}{lgunas} disciplinas, como la música y las matemáticas, 
han recurrido al lenguaje artificial o simbólico para simplificar la expresión de ideas complejas. A diferencia de los 
lenguajes naturales como el español, el inglés o el ciguayo, que son herramientas generales de comunicación, los 
lenguajes simbólicos están diseñados con un propósito específico, lo que nos permite expresar pensamientos en términos 
precisos sin ambigüedad y contenido emocional.



\textbf{Forma de usar el ambiente del ejemplo con nombre del ejemplo. Recordar que cada ambiente tiene dos copciones, 
la primero es para el nombre y la segunda para la etiqueta de referencia.}
\begin{ejem}[\;\;(Negación)][ejem:02:03]
\centering
La función seno es periódica.

\smallskip

La función seno \textcolor{Naranja}{\textbf{no}} es periódica.
\end{ejem}


%-----------------------------------------
\subsection{\textit{Título de la subsección}}

\textbf{Ejemplo de una tabla con encabezado gris}

\begin{table}[H]
\begin{tabular}{c|c|c}
\rowcolor{gray!20}$p$ & $q$ & $p\wedge q$\\
\toprule
\textcolor{blue}{\textbf{V}} & \textcolor{blue}{\textbf{V}} & \textcolor{blue}{\textbf{V}}\\
\textcolor{blue}{\textbf{V}} & \textcolor{red}{\textbf{F}} & \textcolor{red}{\textbf{F}}\\
\textcolor{red}{\textbf{F}} & \textcolor{blue}{\textbf{V}} & \textcolor{red}{\textbf{F}}\\
\textcolor{red}{\textbf{F}} & \textcolor{red}{\textbf{F}} & \textcolor{red}{\textbf{F}}\\
\toprule
\end{tabular}
\caption{Tabla de verdad para la conjunción}\label{tab:02:03}
\end{table}

\textbf{Ejemplo de una tabla con encabezado azul claro}
\begin{table}[H]
\begin{tabular}{c|c|c|c|c|c}
\rowcolor{LightBlue2}$p$ & $q$ & $p\wedge q$ & $p\longrightarrow q$ & $\neg(p\longrightarrow q)$ & $(p\wedge 
q)\vee\neg(p\longrightarrow q)$\\
\toprule
\textcolor{red}{\textbf{F}} & \textcolor{red}{\textbf{F}} & \textcolor{red}{\textbf{F}} & \textcolor{blue}{\textbf{V}} & 
\textcolor{red}{\textbf{F}} & \textcolor{red}{\textbf{F}}\\
\textcolor{red}{\textbf{F}} & \textcolor{blue}{\textbf{V}} & \textcolor{red}{\textbf{F}} & \textcolor{blue}{\textbf{V}} 
& \textcolor{red}{\textbf{F}} & \textcolor{red}{\textbf{F}}\\
\textcolor{blue}{\textbf{V}} & \textcolor{red}{\textbf{F}} & \textcolor{red}{\textbf{F}} & \textcolor{red}{\textbf{F}} & 
\textcolor{blue}{\textbf{V}} & \textcolor{blue}{\textbf{V}}\\
\textcolor{blue}{\textbf{V}} & \textcolor{blue}{\textbf{V}} & \textcolor{blue}{\textbf{V}} & 
\textcolor{blue}{\textbf{V}} & \textcolor{red}{\textbf{F}} & \textcolor{blue}{\textbf{V}}\\ 
\toprule
\end{tabular}
\caption{Tabla del ejemplo \ref{ejem:02:46}}\label{tab:02:07}
\end{table}


%-----------------------------------------
\cleardoublepage
\subchapter{Problemas propuestos}
\setenumerate[1]{label=\bfseries{\alph*)\quad}, labelindent=\parindent}
\setenumerate[2]{label=\bfseries\arabic*.}
\setenumerate[3]{label=\bfseries{\roman*})}
\pagestyle{probprop}
\pagecolor{paginaprob}

\begin{prob} ¿Cuáles de las siguientes son proposiciones? En caso que sea un proposición, diga si es verdadera o falsa.
\begin{multicols}{2}
\begin{enumerate}
     \item $7-4=3$.
     \item $5^{4}<3^{2}$.
     \item En la escuela nos enseñan a caminar.
     \item Tu voto es tu opinión.
     \item ¿Te duele?
     \item !`Cállate la boca!
     \item Aquel árbol es azul.
     \item La música es una expresión del arte.
\end{enumerate}
\end{multicols}
\end{prob}


%-----------------------------------------------------------------------
% Capítulo 3
%-----------------------------------------------------------------------
\chapter{Título del capítulo}\label{cap:03}
\pagecolor{white}
\BgThispage
\thispagestyle{empty}
\pagestyle{isfodosuCD}
\epigraph{La libertad está en ser dueños de nuestra propia vida.}{Platón}

\lettrine[lraise=0, lines=4, loversize=0]{\textcolor{azulF}{P}}{ara} la lógica de proposiciones, la \textit{lógica 
formal} puede determinar la validez de cualquier razonamiento, donde la proposición analizada se deriva de una 
proposición sin analizar de otra u otras proposiciones que no se analizaban. En otras palabras: la lógica formal, a 
nivel de lógica de proposiciones, solo puede analizar formalmente de manera completa aquellos argumentos cuya estructura 
interna de las proposiciones que las componen en su validez no tiene ningún papel.

%-----------------------------------------
% \cleardoublepage
\subchapter{Problemas propuestos}
\pagestyle{probprop}
\pagecolor{paginaprob}

%-----------------------------------------------------------------------
% Capítulo 4
%-----------------------------------------------------------------------
\chapter{Título del capítulo}\label{cap:04}
\pagecolor{white}
\BgThispage
\thispagestyle{empty}
\pagestyle{isfodosuCD}
\epigraph{Si no esperas lo inesperado no lo reconocerás cuando llegue.}{Heráclito}

\lettrine[lraise=0, lines=4, loversize=0]{\textcolor{azulF}{A}}{unque} la \textbf{teoría de conjuntos} se reconoce como 
la piedra angular de las \textit{nuevas matemáticas}, el pensamiento intuitivo de los conjuntos no es esencialmente 
nada nuevo. Los matemáticos han considerado colecciones de un tipo u otro desde los tiempos más remotos, y los 
conceptos básicos de la teoría de conjuntos moderna están implícitos en un gran número de argumentos clásicos. Sin 
embargo, no fue hasta finales del siglo $XIX$, con la obra de \textcolor{azulF}{\textbf{George Cantor ($1845-1918$)}}, 
que los conjuntos se convirtieron en el principal objeto de la teoría matemática.

\textbf{ejemplo del ambiente colorario}

\begin{cor}[][coro:04:01] Si $A$ y $B$ son conjuntos, las siguientes afirmaciones son equivalentes entre sí, es decir, 
siempre que alguna de ellas sea verdadera, entonces las demás lo serán:
\begin{enumerate}[label=\itembolasgrises{\arabic{*}}]
 \item\label{corolario:04:01:01} $A=B$;
 \item\label{corolario:04:01:02} $A$ y $B$ tienen los mismos elementos;
 \item\label{corolario:04:01:03} $A\subseteq B$ y $B\subseteq A$;
 \item\label{corolario:04:01:04} $(\forall\,\,x:\,\,x\in A\longrightarrow x\in B)$ y $(\forall\,\,x:\,\,x\in B\longrightarrow x\in A)$;
 \item\label{corolario:04:01:05} $\forall\,\,x:\,\,x\in A\longleftrightarrow x\in B$;
 \item\label{corolario:04:01:06} para todo elemento $x$, o bien $x\in A$ y $x\in B$, o bien $x\notin A$ y $x\notin B$.
\end{enumerate}
\end{cor}

\newpage

\textbf{Ejemplo del ambiente teorema}

\begin{teo}[][teo:04:01] Sea $\mathcal{U}$ el conjunto universal y $A$, $B$, y $C$ tres conjuntos. Entonces se satisface las siguientes afirmaciones:
\begin{enumerate}[label=\itembolasgrises{\arabic{*}}]
 \item\label{teorema:04:01:01} $\emptyset\subseteq A$.
 \item\label{teorema:04:01:02} $A\subseteq A$.
 \item\label{teorema:04:01:03} Si $A\subseteq B$ y $B\subseteq C$, entonces $A\subseteq C$.
 \item\label{teorema:04:01:04} Si $A\subseteq B$ y $B\subseteq A$, entonces $A=B$.
\end{enumerate}
\end{teo}

\textbf{ejemplo del ambiente para demostraciones}
\begin{proof} Para demostrar el apartado \ref{teorema:04:01:01}, hagamos una demostración por reducción al absurdo, o 
también conocida como el método indirecto. Supongamos que $\emptyset\nsubseteq A$, es decir, existe $x\in\emptyset$ y 
$x\notin A$. Ahora, como $\emptyset$ no tiene elementos, entonces $x$ no puede estar en el conjunto $\emptyset$, por lo 
tanto lo supuesto es falso, y entonces se tiene que $\emptyset\subseteq A$.
\end{proof}


\begin{propo}[][pro:04:01] Sean $A$, $B$ y $C$ tres conjuntos sobre un conjunto unversal $\mathcal{U}$, entonces se tiene:
\begin{enumerate}[label=\itembolasgrises{\arabic{*}}]
 \item\label{proposicion:04:01:01} $A\cup A=A$ y $A\cap A=A$ (idempotencia).
 \item\label{proposicion:04:01:02} $A\cup B=B\cup A$ y $A\cap B=B\cap A$ (conmutativa).
 \item\label{proposicion:04:01:03} $A\subseteq A\cup B$, $B\subseteq A\cup B$, $A\cap B\subseteq A$, y $A\cap B\subseteq B$ (absorción).
 \item\label{proposicion:04:01:04} $(A\cup B)\cup C=A\cup(B\cup C)$ y $(A\cap B)\cap C=A\cap(B\cap C)$ (asociativa).
\end{enumerate}
\end{propo}


%-----------------------------------------
% \cleardoublepage
\subchapter{Problemas propuestos}
\pagestyle{probprop}
\pagecolor{paginaprob}


%-----------------------------------------------------------------------
% Capítulo 5
%-----------------------------------------------------------------------
\chapter{Título del capítulo}\label{cap:05}
\pagecolor{white}
\BgThispage
\thispagestyle{empty}
\pagestyle{isfodosuCD}
\epigraph{Se mide la inteligencia de un individuo por la cantidad de incertidumbres que es capaz de soportar.}{Immanuel 
Kant}

\lettrine[lraise=0, lines=4, loversize=0]{\textcolor{azulF}{E}}{n} este capítulo desarrollaremos el concepto de 
\textit{relaciones entre conjuntos}. Este es un concepto importante en matemáticas. Para nosotros, su mayor utilidad es 
que se basa en la definición de funciones que veremos más adelante. Ya estamos familiarizados con las relaciones, por 
ejemplo: $a=b$, la relación de \textit{igualdad}; $a<b$, la relación de \textit{menor que}; $X\subseteq Y$ la relación 
de subconjunto; $m|n$ la relación de \textit{divisor de}, y así muchas otras.

%-----------------------------------------
% \cleardoublepage
\subchapter{Problemas propuestos}
% \setenumerate[1]{label=\bfseries{\alph*)\quad}, labelindent=\parindent}
% \setenumerate[2]{label=\bfseries\arabic*.}
% \setenumerate[3]{label=\bfseries{\roman*})}
\pagestyle{probprop}
\pagecolor{paginaprob}


%-----------------------------------------------------------------------
% Capítulo 6
%-----------------------------------------------------------------------
\chapter{Título del capítulo}\label{cap:06}
\pagecolor{white}
\BgThispage
\thispagestyle{empty}
\pagestyle{isfodosuCD}
\epigraph{Acusar a los demás de los infortunios propios es un signo de falta de educación. Acusarse a uno mismo 
demuestra que la educación ha comenzado.}{Epicteto}


\lettrine[lraise=0, lines=4, loversize=0]{\textcolor{azulF}{E}}{l} concepto de \textit{función} es una de las ideas 
matemáticas más básicas, y entra en casi todas las discusiones de la matemática, de ella se hace uso del concepto de 
función. El concepto de correspondencia entre conjuntos o de relación entre conjuntos consiste en el que, dados dos 
conjuntos $A$ y $B$, se define un subconjunto $R$ de $A\times B$, lo cual es muy general. Si se puede decir de cierta 
manera, este tipo de relaciones son ambiguas.


%-----------------------------------------
\subchapter{Problemas propuestos}
% \setenumerate[1]{label=\bfseries{\alph*)\quad}, labelindent=\parindent}
% \setenumerate[2]{label=\bfseries\arabic*.}
% \setenumerate[3]{label=\bfseries{\roman*})}
\pagestyle{probprop}
\pagecolor{paginaprob}



%-----------------------------------------------------------------------
% Capítulo 7
%-----------------------------------------------------------------------
\chapter{Título del Capítulo}\label{cap:07}
\pagecolor{white}
\BgThispage
\thispagestyle{empty}
\pagestyle{isfodosuCD}
\epigraph{Los educadores, más que cualquier otra clase de profesionales, son los guardianes de la 
civilización.}{Bertrand Russell}

\lettrine[lraise=0, lines=4, loversize=0]{\textcolor{azulF}{E}}{n} este capítulo trataremos la numerabilidad o no de 
conjuntos, en particular se demostrará la numerabilidad de $\mathbb{Q}$ y la no numerabilidad de $\mathbb{R}$. Pero 
antes debemos introducir algunos conceptos previos sobre la cardinalidad de conjuntos. El tamaño de un conjunto finito 
puede medirse fácilmente; por ejemplo, el tamaño del conjunto $A=\{1,2,3,\ldots,50\}$ es $50$ porque tiene $50$ 
elementos, y el tamaño de los conjuntos $B=\{\pi,2,30,-2\}$ y $C=\{9,11,-1,5\}$ es $4$.

\textbf{ejemplo del ambiente para Lemas}

\begin{lma}[][lema:07:01] El conjunto de números racionales positivos, $\mathbb{Q}^{+}$, es numerable.
\end{lma}

%-----------------------------------------
\subchapter{Problemas propuestos}
% \setenumerate[1]{label=\bfseries{\alph*)\quad}, labelindent=\parindent}
% \setenumerate[2]{label=\bfseries\arabic*.}
% \setenumerate[3]{label=\bfseries{\roman*})}
\pagestyle{probprop}
\pagecolor{paginaprob}



%:Empezamos con los apéndices, que irían en uno o más ficheros. Es necesario incluir estos ficheros entre el entorno \begin{appendices}....\end{appendices} debido a que se ha deseado utilizar un formato diferente para el título de los apéndices, incluyendo la palabra apéndice, para la numeración de los apéndices, alfabético, y para las cabeceras de las páginas.

% \begin{appendices}
% % Fichero en el que se incluyen los apéndices
% \include{apendices/apendices} %Ver este fichero para incluir ahí los apéndices.
% \end{appendices}
%:Fin de la inclusión de apéndices

%:Empieza todo lo que no constituye el cuerpo en si del libro. Todo lo que va detrás

\backmatter

%:Bibliografía con biblatex y biber
\cleardoublepage
\phantomsection
\addcontentsline{toc}{listasb}{\bibname}
\pagestyle{especial}
\pagecolor{white}
%BIBER
%\printbibliography[heading=etsi]
%BIBTEX
% \bibliographystyle{IEEEtran}
\bibliographystyle{amsplain} %flexbib amsplain alpha
%:Fichero con la bibliografía, BIBTEX
\bibliography{bibliografiaLibroISFODOSU}

%:Índice alfabético
% \cleardoublepage
% \phantomsection
% \addcontentsline{toc}{listasb}{\indexname}
% \chaptermark{\indexname}
% \printindex

%:Acrónimos
% \cleardoublepage
% \phantomsection
% \addcontentsline{toc}{listasb}{\glossaryname}
% \chaptermark{\glossaryname}
% \printglossaries

\end{document}
