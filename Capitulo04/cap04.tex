%-----------------------------------------------------------------------
% Capítulo 4
%-----------------------------------------------------------------------
\chapter{Título del capítulo}\label{cap:04}
\pagecolor{white}
\BgThispage
\thispagestyle{empty}
\pagestyle{isfodosuCD}
\epigraph{Si no esperas lo inesperado no lo reconocerás cuando llegue.}{Heráclito}

\lettrine[lraise=0, lines=4, loversize=0]{\textcolor{azulF}{A}}{unque} la \textbf{teoría de conjuntos} se reconoce como 
la piedra angular de las \textit{nuevas matemáticas}, el pensamiento intuitivo de los conjuntos no es esencialmente 
nada nuevo. Los matemáticos han considerado colecciones de un tipo u otro desde los tiempos más remotos, y los 
conceptos básicos de la teoría de conjuntos moderna están implícitos en un gran número de argumentos clásicos. Sin 
embargo, no fue hasta finales del siglo $XIX$, con la obra de \textcolor{azulF}{\textbf{George Cantor ($1845-1918$)}}, 
que los conjuntos se convirtieron en el principal objeto de la teoría matemática.

\textbf{ejemplo del ambiente colorario}

\begin{cor}[][coro:04:01] Si $A$ y $B$ son conjuntos, las siguientes afirmaciones son equivalentes entre sí, es decir, 
siempre que alguna de ellas sea verdadera, entonces las demás lo serán:
\begin{enumerate}[label=\itembolasgrises{\arabic{*}}]
 \item\label{corolario:04:01:01} $A=B$;
 \item\label{corolario:04:01:02} $A$ y $B$ tienen los mismos elementos;
 \item\label{corolario:04:01:03} $A\subseteq B$ y $B\subseteq A$;
 \item\label{corolario:04:01:04} $(\forall\,\,x:\,\,x\in A\longrightarrow x\in B)$ y $(\forall\,\,x:\,\,x\in B\longrightarrow x\in A)$;
 \item\label{corolario:04:01:05} $\forall\,\,x:\,\,x\in A\longleftrightarrow x\in B$;
 \item\label{corolario:04:01:06} para todo elemento $x$, o bien $x\in A$ y $x\in B$, o bien $x\notin A$ y $x\notin B$.
\end{enumerate}
\end{cor}

\newpage

\textbf{Ejemplo del ambiente teorema}

\begin{teo}[][teo:04:01] Sea $\mathcal{U}$ el conjunto universal y $A$, $B$, y $C$ tres conjuntos. Entonces se satisface las siguientes afirmaciones:
\begin{enumerate}[label=\itembolasgrises{\arabic{*}}]
 \item\label{teorema:04:01:01} $\emptyset\subseteq A$.
 \item\label{teorema:04:01:02} $A\subseteq A$.
 \item\label{teorema:04:01:03} Si $A\subseteq B$ y $B\subseteq C$, entonces $A\subseteq C$.
 \item\label{teorema:04:01:04} Si $A\subseteq B$ y $B\subseteq A$, entonces $A=B$.
\end{enumerate}
\end{teo}

\textbf{ejemplo del ambiente para demostraciones}
\begin{proof} Para demostrar el apartado \ref{teorema:04:01:01}, hagamos una demostración por reducción al absurdo, o 
también conocida como el método indirecto. Supongamos que $\emptyset\nsubseteq A$, es decir, existe $x\in\emptyset$ y 
$x\notin A$. Ahora, como $\emptyset$ no tiene elementos, entonces $x$ no puede estar en el conjunto $\emptyset$, por lo 
tanto lo supuesto es falso, y entonces se tiene que $\emptyset\subseteq A$.
\end{proof}


\begin{propo}[][pro:04:01] Sean $A$, $B$ y $C$ tres conjuntos sobre un conjunto unversal $\mathcal{U}$, entonces se tiene:
\begin{enumerate}[label=\itembolasgrises{\arabic{*}}]
 \item\label{proposicion:04:01:01} $A\cup A=A$ y $A\cap A=A$ (idempotencia).
 \item\label{proposicion:04:01:02} $A\cup B=B\cup A$ y $A\cap B=B\cap A$ (conmutativa).
 \item\label{proposicion:04:01:03} $A\subseteq A\cup B$, $B\subseteq A\cup B$, $A\cap B\subseteq A$, y $A\cap B\subseteq B$ (absorción).
 \item\label{proposicion:04:01:04} $(A\cup B)\cup C=A\cup(B\cup C)$ y $(A\cap B)\cap C=A\cap(B\cap C)$ (asociativa).
\end{enumerate}
\end{propo}


%-----------------------------------------
% \cleardoublepage
\subchapter{Problemas propuestos}
\pagestyle{probprop}
\pagecolor{paginaprob}

